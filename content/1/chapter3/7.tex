后端的任务是从模块的IR中创建优化的机器码。IR是后端的接口,可以使用C++接口或文本形式创建。同样,IR由AST生成。\par

\hspace*{\fill} \par %插入空行
\textbf{LLVM IR的文本表示}

尝试生成LLVM IR之前,需要了解我们想要生成的东西。对于表达式语言的例子,计划如下。

\begin{enumerate}
\item 询问每个变量的值。
\item 计算表达式的值。
\item 打印结果。
\end{enumerate}

为了要求用户提供一个变量的值并打印出结果,使用了两个库函数,calc\underline{~}read()和calc\underline{~}write()。对于\textit{with a: 3*a}表达式,生成的IR如下:\par

\begin{enumerate}
\item 标准库函数必须声明,语法类似于C。函数名之前的类型是返回类型。由圆括号包围的类型名是参数类型。声明可以出现在文件的任何地方:
\begin{tcolorbox}[colback=white,colframe=black]
declare i32 @calc\underline{~}read(i8*) \\
declare void @calc\underline{~}write(i32)
\end{tcolorbox}

\item calc\underline{~}read()函数将变量名作为参数。下面的定义了一个常数,持有a和空字节,空字节在C语言中作为字符串的结束符。
\begin{tcolorbox}[colback=white,colframe=black]
@a.str = private constant [2 x i8] c"a$\setminus$00"
\end{tcolorbox}

\item 接下来是main()函数。省略了参数的名字,因为没使用到。像C语言一样,函数的主体置于大括号中。
\begin{tcolorbox}[colback=white,colframe=black]
define i32 @main(i32, i8**) \{
\end{tcolorbox}

\item 每个基本块都必须有一个标签。因为这是函数的第一个基本块,我们把它命名为entry。
\begin{tcolorbox}[colback=white,colframe=black]
entry:
\end{tcolorbox}

\item 调用calc\underline{~}read()函数来读取a变量的值。嵌套使用getelemenptr指令进行了索引计算,以计算指向字符串常量第一个元素的指针。该函数的结果分配给未命名的\%2变量。
\begin{tcolorbox}[colback=white,colframe=black]
\hspace*{1cm}\%2 = call i32 @calc\underline{~}read(i8* getelementptr inbounds \\
\hspace*{3.5cm}([2 x i8], [2 x i8]* @a.str, i32 0, i32
0))
\end{tcolorbox}

\item 接下来,该变量乘以3。
\begin{tcolorbox}[colback=white,colframe=black]
\hspace*{1cm}\%3 = mul nsw i32 3, \%2
\end{tcolorbox}

\item 结果通过calc\underline{~}write()函数打印结果到控制台。
\begin{tcolorbox}[colback=white,colframe=black]
	\hspace*{1cm}call void @calc\underline{~}write(i32 \%3)
\end{tcolorbox}

\item 最后,main()函数返回0,表示成功执行。
\begin{tcolorbox}[colback=white,colframe=black]
	\hspace*{1cm}ret i32 0 \\
\}
\end{tcolorbox}

\end{enumerate}

LLVM IR中的每个值都是类型化的,i32表示32位的整数类型,i8*表示一个字节的指针。IR代码非常易读(除了getelementptr操作,这将在第5章,IR生成的基础知识中详细解释)。既然现在已经清楚了IR的样子,就使用AST生成它吧。\par

\hspace*{\fill} \par %插入空行
\textbf{使用AST生成IR}

该接口在CodeGen.h头文件中提供:

\begin{lstlisting}[caption={}]
#ifndef CODEGEN_H
#define CODEGEN_H

#include "AST.h"

class CodeGen
{
public:
	void compile(AST *Tree);
};
#endif
\end{lstlisting}

因为AST包含了语义分析阶段的信息,所以基本思路是用访问者来遍历AST。CodeGen.cpp文件的实现方式如下:

\begin{enumerate}
\item 所需的包括在文件的顶端:
\begin{lstlisting}[caption={}]
#include "CodeGen.h"
#include "llvm/ADT/StringMap.h"
#include "llvm/IR/IRBuilder.h"
#include "llvm/IR/LLVMContext.h"
#include "llvm/Support/raw_ostream.h"
\end{lstlisting}

\item LLVM库的命名空间用于查找名称:
\begin{lstlisting}[caption={}]
using namespace llvm;
\end{lstlisting}

\item 首先,在访问器中声明一些私有成员。每个编译单元在LLVM中由Module类表示,访问者有一个指向模块调用的指针M。为了便于生成IR,使用了Builder\ (其类型为IRBuilder<>)。LLVM用类的层次结构来表示IR中的类型,可以在LLVM的上下文中查找基本类型的实例,如i32。这些基本类型会经常使用。为了避免重复查找,需要缓存类型实例,可以是VoidTy、Int32Ty、Int8PtrTy、Int8PtrPtrTy或Int32Zero。V是当前的计算值,通过树形遍历更新。最后,nameMap将一个变量名映射为由calc\underline{~}read()函数返回的值:
\begin{lstlisting}[caption={}]
namespace {
class ToIRVisitor : public ASTVisitor {
	Module *M;
	IRBuilder<> Builder;
	Type *VoidTy;
	Type *Int32Ty;
	Type *Int8PtrTy;
	Type *Int8PtrPtrTy;
	Constant *Int32Zero;
	Value *V;
	StringMap<Value *> nameMap;
\end{lstlisting}

\item 构造函数初始化了所有成员:
\begin{lstlisting}[caption={}]
public:
	ToIRVisitor(Module *M) : M(M), Builder(M->getContext())
	{
		VoidTy = Type::getVoidTy(M->getContext());
		Int32Ty = Type::getInt32Ty(M->getContext());
		Int8PtrTy = Type::getInt8PtrTy(M->getContext());
		Int8PtrPtrTy = Int8PtrTy->getPointerTo();
		Int32Zero = ConstantInt::get(Int32Ty, 0, true);
	}
\end{lstlisting}

\item 对于每个函数,都必须创建一个FunctionType实例。在C++中,这就是一个函数原型。函数本身是用function实例定义的。首先,run()方法定义了LLVM IR中的main()函数:
\begin{lstlisting}[caption={}]
	void run(AST *Tree) {
		FunctionType *MainFty = FunctionType::get(
			Int32Ty, {Int32Ty, Int8PtrPtrTy}, false);
		Function *MainFn = Function::Create(
			MainFty, GlobalValue::ExternalLinkage,
			"main", M);
\end{lstlisting}

\item 然后,创建带有入口标签的BB基本块,并将其附加到IR构建器上。
\begin{lstlisting}[caption={}]
		BasicBlock *BB = BasicBlock::Create(M->getContext(),
											"entry", MainFn);
		Builder.SetInsertPoint(BB);
\end{lstlisting}

\item 做好这些准备工作后,就可以开始进行树形遍历。
\begin{lstlisting}[caption={}]
		Tree->accept(*this);
\end{lstlisting}

\item 遍历树后,通过calc\underline{~}write()函数打印计算值。同样,必须创建函数原型(FunctionType的实例)。唯一的参数是当前值V:
\begin{lstlisting}[caption={}]
		FunctionType *CalcWriteFnTy =
			FunctionType::get(VoidTy, {Int32Ty}, false);
		Function *CalcWriteFn = Function::Create(
			CalcWriteFnTy, GlobalValue::ExternalLinkage,
			"calc_write", M);
		Builder.CreateCall(CalcWriteFnTy, CalcWriteFn, {V});
\end{lstlisting}

\item 生成结束时,main()函数返回0:
\begin{lstlisting}[caption={}]
		Builder.CreateRet(Int32Zero);
	}
\end{lstlisting}

\item WithDecl节点保存声明变量的名称。首先,需要为calc\underline{~}read()函数创建一个函数原型:
\begin{lstlisting}[caption={}]
	virtual void visit(WithDecl &Node) override {
		FunctionType *ReadFty =
			FunctionType::get(Int32Ty, {Int8PtrTy}, false);
		Function *ReadFn = Function::Create(
			ReadFty, GlobalValue::ExternalLinkage,
			"calc_read", M);
\end{lstlisting}

\item 该方法遍历变量名:
\begin{lstlisting}[caption={}]
		for (auto I = Node.begin(), E = Node.end(); I != E;
			++I) {
\end{lstlisting}

\item 对于每个变量,都会创建一个带有变量名称的字符串。
\begin{lstlisting}[caption={}]
			StringRef Var = *I;
			Constant *StrText = ConstantDataArray::getString(
				M->getContext(), Var);
			GlobalVariable *Str = new GlobalVariable(
				*M, StrText->getType(),
				/*isConstant=*/true,
				GlobalValue::PrivateLinkage,
				StrText, Twine(Var).concat(".str"));
\end{lstlisting}

\item 然后,创建调用calc\underline{~}read()函数的IR代码。在上一步骤中创建的字符串会作为参数传递:
\begin{lstlisting}[caption={}]
			Value *Ptr = Builder.CreateInBoundsGEP(
				Str, {Int32Zero, Int32Zero}, "ptr");
			CallInst *Call =
				Builder.CreateCall(ReadFty, ReadFn, {Ptr});
\end{lstlisting}

\item 返回值存储在mapNames中:
\begin{lstlisting}[caption={}]
			nameMap[Var] = Call;
		}
\end{lstlisting}

\item 使用树形遍历继续遍历表达式:
\begin{lstlisting}[caption={}]
		Node.getExpr()->accept(*this);
	};
\end{lstlisting}

\item Factor节点可以是变量名,也可以是数字。对于变量名,将在mapNames中查找值。对于数字,该值转换为常数值:
\begin{lstlisting}[caption={}]
	virtual void visit(Factor &Node) override {
		if (Node.getKind() == Factor::Ident) {
			V = nameMap[Node.getVal()];
		} else {
			int intval;
			Node.getVal().getAsInteger(10, intval);
			V = ConstantInt::get(Int32Ty, intval, true);
		}
	};
\end{lstlisting}

\item 最后,对于BinaryOp节点,必须使用正确的计算操作:
\begin{lstlisting}[caption={}]
	virtual void visit(BinaryOp &Node) override {
		Node.getLeft()->accept(*this);
		Value *Left = V;
		Node.getRight()->accept(*this);
		Value *Right = V;
		switch (Node.getOperator()) {
			case BinaryOp::Plus:
			V = Builder.CreateNSWAdd(Left, Right); break;
			case BinaryOp::Minus:
			V = Builder.CreateNSWSub(Left, Right); break;
			case BinaryOp::Mul:
			V = Builder.CreateNSWMul(Left, Right); break;
			case BinaryOp::Div:
			V = Builder.CreateSDiv(Left, Right); break;
		}
	};
};
}
\end{lstlisting}

\item 这样,访问者类就完成了。compile()方法创建了全局上下文和模块,执行了树形遍历,并将生成的IR转储到控制台:
\begin{lstlisting}[caption={}]
void CodeGen::compile(AST *Tree) {
	LLVMContext Ctx;
	Module *M = new Module("calc.expr", Ctx);
	ToIRVisitor ToIR(M);
	ToIR.run(Tree);
	M->print(outs(), nullptr);
}
\end{lstlisting}
\end{enumerate}

我们实现了编译器的前端,从读取源代码到生成IR。当然,所有这些组件必须在用户输入时协同工作,这是编译器驱动程序的任务,现在还需要实现运行时所需的功能。我们将在下一节中讨论这两个问题。\par

\hspace*{\fill} \par %插入空行
\textbf{驱动程序和运行时库}

前几节中的所有阶段都是由Calc.cpp驱动程序合在一起的,我们将在这里实现它。需要为输入表达式声明一个参数,初始化LLVM,调用前几节中的所有阶段。让我们来看看:

\begin{enumerate}
\item 首先,必须包含必需的头文件:
\begin{lstlisting}[caption={}]
#include "CodeGen.h"
#include "Parser.h"
#include "Sema.h"
#include "llvm/Support/CommandLine.h"
#include "llvm/Support/InitLLVM.h"
#include "llvm/Support/raw_ostream.h"
\end{lstlisting}

\item LLVM自带了声明命令行选项的系统,只需要为所需的每个选项声明一个静态变量。在此过程中,该选项将被注册到全局命令行解析器中。这种方法的优点是,每个组件都可以在需要时添加命令行选项。我们必须为输入表达式声明一个选项:
\begin{lstlisting}[caption={}]
static llvm::cl::opt<std::string>
	Input(llvm::cl::Positional,
		llvm::cl::desc("<input expression>"),
		llvm::cl::init(""));
\end{lstlisting}

\item 在main()函数中,LLVM库初始化,需要调用ParseCommandLineOptions来处理命令行上的选项。这也可以处理打印帮助信息。在出现错误的情况下,此方法会让应用程序终止运行:
\begin{lstlisting}[caption={}]
int main(int argc, const char **argv) {
	llvm::InitLLVM X(argc, argv);
	llvm::cl::ParseCommandLineOptions(
		argc, argv, "calc - the expression compiler\n");
\end{lstlisting}

\item 接下来,使用词法分析器和解析器。语法分析之后,检查是否发生了错误。如果有错,则使用返回代码退出编译器,表示失败:
\begin{lstlisting}[caption={}]
	Lexer Lex(Input);
	Parser Parser(Lex);
	AST *Tree = Parser.parse();
	if (!Tree || Parser.hasError()) {
		llvm::errs() << "Syntax errors occured\n";
		return 1;
	}
\end{lstlisting}

\item 如果有语义错误,也是一样:
\begin{lstlisting}[caption={}]
	Sema Semantic;
	if (Semantic.semantic(Tree)) {
		llvm::errs() << "Semantic errors occured\n";
		return 1;
	}
\end{lstlisting}

\item 最后,在驱动程序中,调用代码生成器:
\begin{lstlisting}[caption={}]
	CodeGen CodeGenerator;
	CodeGenerator.compile(Tree);
	return 0;
}
\end{lstlisting}

\end{enumerate}

这样,就成功地为用户输入创建了IR代码。我们将目标代码的生成委托给LLVM静态编译器llc,这样就完成了编译器的实现。我们必须将所有组件链接到一起,从而创建calc应用程序。\par

运行库由一个名为rtcalc.c的文件组成。它包含了用C语言编写的calc\underline{~}read()和calc\underline{~}write()函数的实现。\par

\begin{lstlisting}[caption={}]
#include <stdio.h>
#include <stdlib.h>

void calc_write(int v)
{
	printf("The result is: %d\n", v);
}
\end{lstlisting}

calc\underline{~}write()只把结果值输出到终端。

\begin{lstlisting}[caption={}]
int calc_read(char *s)
{
	char buf[64];
	int val;
	printf("Enter a value for %s: ", s);
	fgets(buf, sizeof(buf), stdin);
	if (EOF == sscanf(buf, "%d", &val))
	{
		printf("Value %s is invalid\n", buf);
		exit(1);
	}
	return val;
}
\end{lstlisting}

calc\underline{~}read()从终端读取一个整数。用户可以有任意的输入,所以我们必须仔细检查输入。如果输入不是数字,则应用程序退出。一种更复杂的方法是让用户意识到这个问题,并再次请求输入数字。\par

现在,我们可以试试我们的编译器。calc应用程序从一个表达式创建IR。LLVM静态编译器将IR编译为一个目标文件。然后,可以使用您喜欢的C编译器链接到小型运行时库。在Unix上,可以这样:\par

\begin{tcolorbox}[colback=white,colframe=black]
\$ calc "with a: a*3" | llc –filetype=obj –o=expr.o \\
\$ clang –o expr expr.o rtcalc.c \\
\$ expr \\
\hspace*{0.5cm}Enter a value for a: 4 \\
\hspace*{0.5cm}The result is: 12
\end{tcolorbox}

在Windows上,很可能会使用cl编译器:\par

\begin{tcolorbox}[colback=white,colframe=black]
\$ calc "with a: a*3" | llc –filetype=obj –o=expr.obj \\
\$ cl expr.obj rtcalc.c \\
\$ expr \\
\hspace*{0.5cm}Enter a value for a: 4 \\
\hspace*{0.5cm}The result is: 12
\end{tcolorbox}

这样,就创建了第一个基于llvm的编译器!请花点时间研究一下这些不同的表达方式。还要检查乘法运算符是否在加法运算符之前求值,以及使用括号是否会改变求值顺序,这与基本计算器的要求一致。\par







