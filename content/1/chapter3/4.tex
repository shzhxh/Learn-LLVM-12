正如上一节的示例中所看到的,编程语言由许多元素组成,例如:关键字、标识符、数字、操作符等。词法分析的任务是获取文本输入并从中创建标记序列。calc语言由with、:、+、-、*、/、(、)、([a-zA-Z])+ (标识符) 和 ([0-9])+ (数字) 正则表达式组成。为了方便处理,为每个标记分配了索引(数字)。\par

\hspace*{\fill} \par %插入空行 
\textbf{手写的词法分析器}

词法分析程序的实现通常称为Lexer。创建一个名为Lexer.h的头文件,并开始定义Token:\par

\begin{lstlisting}[caption={}]
#ifndef LEXER_H
#define LEXER_H

#include "llvm/ADT/StringRef.h"
#include "llvm/Support/MemoryBuffer.h"
\end{lstlisting}

llvm::MemoryBuffer类提供对一个内存块的只读访问,该内存块中填充了文件的内容。请求时,末尾的零字符('$\setminus$x00')添加到缓冲区的末尾。使用这个特性来读取整个缓冲区,而不检查每次访问缓冲区的长度。llvm::StringRef类封装了一个指向C字符串及其长度的指针。因为长度已存储的,所以字符串不需要像普通的C字符串那样以0字符结束('$\setminus$x00')。这允许StringRef的实例指向MemoryBuffer管理的内存。让我们更详细地了解一下:\par

\begin{enumerate}
\item 首先,Tonken类包含前面提到的token编号(枚举):
\begin{lstlisting}[caption={}]
class Lexer;

class Token {
	friend class Lexer;
	
public:
	enum TokenKind : unsigned short {
		eoi, unknown, ident, number, comma, colon, plus,
		minus, star, slash, l_paren, r_paren, KW_with
	};
\end{lstlisting}
除了为每个标记定义数字外,我们添加了两个额外的值:eoi和unknown。eoi表示输入的结束,如果处理完输入的所有字符,则返回eoi。unknown用于词汇错误的情况,例如:\#不是语言标识,所以它会映射为unknown。 
	
\item 除了枚举之外,该类还有一个成员Text,指向标记的文本的开头(使用前面提到的StringRef类):
\begin{lstlisting}[caption={}]
private:
	TokenKind Kind;
	llvm::StringRef Text;

public:
	TokenKind getKind() const { return Kind; }
	llvm::StringRef getText() const { return Text; }
\end{lstlisting}
对于语义处理,知道标识符的名称很重要。
	
\item is()和isOneOf()方法用于测试标记是否属于某种类型。isOneOf()使用可变参数模板,允许可变数量的参数:
\begin{lstlisting}[caption={}]
	bool is(TokenKind K) const { return Kind == K; }
	bool isOneOf(TokenKind K1, TokenKind K2) const {
		return is(K1) || is(K2);
	}
	template <typename... Ts>
	bool isOneOf(TokenKind K1, TokenKind K2, Ts... Ks)
	const {
		return is(K1) || isOneOf(K2, Ks...);
	}
};
\end{lstlisting}

\item Lexer类本身也有一个简单接口:
\begin{lstlisting}[caption={}]
class Lexer {
	const char *BufferStart;
	const char *BufferPtr;
	public:
	Lexer(const llvm::StringRef &Buffer) {
		BufferStart = Buffer.begin();
		BufferPtr = BufferStart;
	}
	void next(Token &token);
	private:
	void formToken(Token &Result, const char *TokEnd,
	Token::TokenKind Kind);
};
#endif
\end{lstlisting}
除了构造函数,公共接口只包含next(),返回下一个标记。该方法的作用类似于迭代器,总会指向下一个可用的标记。该类的唯一成员是指向输入开头和下一个未处理字符的指针,并且假设缓冲区以0结尾(类似于C字符串)。
	
\item 让我们在Lexer.cpp文件中实现Lexer类。从一些辅助函数开始,可以帮助分类字符:
\begin{lstlisting}[caption={}]
#include "Lexer.h"

namespace charinfo {
LLVM_READNONE inline bool isWhitespace(char c) {
	return c == ' ' || c == '\t' || c == '\f' ||
	c == '\v' ||
	c == '\r' || c == '\n';
}

LLVM_READNONE inline bool isDigit(char c) {
	return c >= '0' && c <= '9';
}
LLVM_READNONE inline bool isLetter(char c) {
	return (c >= 'a' && c <= 'z') ||
	(c >= 'A' && c <= 'Z');
}
}
\end{lstlisting}
这些函数让条件判断的可读性更好。
\begin{tcolorbox}[colback=blue!5!white,colframe=blue!75!black,title=Note]
不使用<cctype>标准库头提供的函数,有两个原因。首先,这些函数根据环境中定义的语境来改变行为,例如:如果语境是德语,那么德语umlauts可以归类为字母。这在编译器中通常是不需要的。第二,由于函数的参数类型是int,必须从char类型进行转换。这种转换的结果取决于char是当作有符号还是无符号类型,这会导致可移植性问题。
\end{tcolorbox}

\item 上一节的语法中,我们知道了该语言的所有标记。但是语法没有定义需要忽略的字符,例如:空格或新行只会增加空格,并且经常被忽略。next()方法首先要跳过这些字符:
\begin{lstlisting}[caption={}]
void Lexer::next(Token &token) {
	while (*BufferPtr &&
	charinfo::isWhitespace(*BufferPtr)) {
		++BufferPtr;
	}
\end{lstlisting}

\item 接下来,确保仍然有字符需要处理:
\begin{lstlisting}[caption={}]
	if (!*BufferPtr) {
		token.Kind = Token::eoi;
		return;
	}
\end{lstlisting}
至少要处理一个字符。
	
\item 接下来,首先检查字符是不是字母(小写或大写均可)。这种情况下,标记只能是是标识符或with关键字,要注意标识符的正则表达式也可能匹配到with关键字。常用的解决方案是收集正则表达式匹配的字符(译者注:即这些字符全是字母),并检查字符串是不是with关键字:
\begin{lstlisting}[caption={}]
	if (charinfo::isLetter(*BufferPtr)) {
		const char *end = BufferPtr + 1;
		while (charinfo::isLetter(*end))
			++end;
		llvm::StringRef Name(BufferPtr, end - BufferPtr);
		Token::TokenKind kind =
			Name == "with" ? Token::KW_with : Token::ident;
		formToken(token, end, kind);
		return;
	}
\end{lstlisting}
formToken()方法用于填充标记。

\item 接下来,我们检查数字。下面的代码与前面的代码非常相似:
\begin{lstlisting}[caption={}]
	else if (charinfo::isDigit(*BufferPtr)) {
		const char *end = BufferPtr + 1;
		while (charinfo::isDigit(*end))
			++end;
		formToken(token, end, Token::number);
		return;
	}
\end{lstlisting}

\item 现在,剩下的标记都是特殊的字符串了,这很容易用switch语句来处理。剩下的这些标记都只有一个字符大小,可以使用CASE宏来减少输入:
\begin{lstlisting}[caption={}]
	else {
		switch (*BufferPtr) {
#define CASE(ch, tok) \
case ch: formToken(token, BufferPtr + 1, tok); break
CASE('+', Token::plus);
CASE('-', Token::minus);
CASE('*', Token::star);
CASE('/', Token::slash);
CASE('(', Token::Token::l_paren);
CASE(')', Token::Token::r_paren);
CASE(':', Token::Token::colon);
CASE(',', Token::Token::comma);
#undef CASE
\end{lstlisting}

\item 最后,我们需要检查意料外的字符:
\begin{lstlisting}[caption={}]
		default:
			formToken(token, BufferPtr + 1, Token::unknown);
		}
	return;
	}
}
\end{lstlisting}
现在,只缺少formToken()了。

\item 这个辅助方法填充Token实例的成员,并更新指针以指向下一个未处理字符:
\begin{lstlisting}[caption={}]
void Lexer::formToken(Token &Tok, const char *TokEnd,
                   Token::TokenKind Kind) {
	Tok.Kind = Kind;
	Tok.Text = llvm::StringRef(BufferPtr, TokEnd -
				   BufferPtr);
	BufferPtr = TokEnd;
}
\end{lstlisting}
\end{enumerate}

下一节中,我们将了解如何构造解析器以进行语法分析。\par





































