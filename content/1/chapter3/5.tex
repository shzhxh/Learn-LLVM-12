句法分析由解析器完成,下面我们将实现解析器,它的基础是前面章节中的语法和词法解析代码。解析处理的结果是一个称为抽象语法树(AST)的动态数据结构。AST是输入的一种非常浓缩的表现形式,非常适合进行语义分析。首先,我们将实现解析器。之后,再一起来了解一下AST。\par


\hspace*{\fill} \par %插入空行
\textbf{手写的解析器}

解析器的接口定义在Parser.h头文件中:\par

\begin{lstlisting}[caption={}]
#ifndef PARSER_H
#define PARSER_H

#include "AST.h"
#include "Lexer.h"
#include "llvm/Support/raw_ostream.h"
\end{lstlisting}

AST.h头文件声明了AST的接口,将在后面展示。LLVM的编码指南禁止使用<iostream>库,所以必须包含相同功能的LLVM头文件。有输出错误信息的需求。让我们看看代码的更多细节:\par

\begin{enumerate}
\item 首先,Parser类声明了一些私有成员:
\begin{lstlisting}[caption={}]
class Parser {
	Lexer &Lex;
	Token Tok;
	bool HasError;
\end{lstlisting}
Lex和Tok是上一节中类的实例。Tok存储下一个标记(look-ahead),而Lex对下一个标记进行检索。HasError表示是否检测到错误。

\item 有几个方法用来处理标记:
\begin{lstlisting}[caption={}]
	void error() {
		llvm::errs() << "Unexpected: " << Tok.getText()
		<< "\n";
		HasError = true;
	}
	
	void advance() { Lex.next(Tok); }
	
	bool expect(Token::TokenKind Kind) {
		if (Tok.getKind() != Kind) {
			error();
			return true;
		}
		return false;
	}
	
	bool consume(Token::TokenKind Kind) {
		if (expect(Kind))
		return true;
		advance();
		return false;
	}
\end{lstlisting}
advance()从词法分析器中检索下一个标记。expect()检查look-ahead是否为预期的类型,如果不是则发出错误消息。consume()会检查look-ahead是预期的类型,如果是则检索下一个标记。error()发出错误消息,并将HasError设置为true。\par

\item 对于语法中的每个非终结符(non-terminal),声明一个方法来解析那个规则:
\begin{lstlisting}[caption={}]
	AST *parseCalc();
	Expr *parseExpr();
	Expr *parseTerm();
	Expr *parseFactor();
\end{lstlisting}
\begin{tcolorbox}[colback=blue!5!white,colframe=blue!75!black, title=Note]
ident和number没有方法来解析。对于这些规则,只需返回标记,并替换为相应的标记。
\end{tcolorbox}

\item 接下来是公共接口。构造函数初始化所有成员并从词法解析器中检索第一个标记:
\begin{lstlisting}[caption={}]
	public:
	Parser(Lexer &Lex) : Lex(Lex), HasError(false) {
		advance();
	}
\end{lstlisting}

\item 需要有一个函数来获取错误标志的值:
\begin{lstlisting}[caption={}]
	bool hasError() { return HasError; }
\end{lstlisting}

\item 最后,parse()方法是解析的入口点:
\begin{lstlisting}[caption={}]
	AST *parse();
};
#endif
\end{lstlisting}
	
\end{enumerate}

下一节中,我们将学习如何实现解析器。\par

\hspace*{\fill} \par %插入空行
\textbf{解析器的实现}

深入解析器的实现:\par

\begin{enumerate}
\item 解析器的实现可以在Parser.cpp文件中找到,从parse()方法开始:
\begin{lstlisting}[caption={}]
#include "Parser.h"

AST *Parser::parse() {
	AST *Res = parseCalc();
	expect(Token::eoi);
	return Res;
}
\end{lstlisting}
parse()的主要工作是处理整个输入。还记得第一节中的解析示例吧,添加了一个特殊符号来表示输入的结束。我们会在这里进行检查。

\item parseCalc()方法实现了calc的规则。我们需要仔细研究一下这个方法,因为其它解析方法遵循相同的模式。回顾一下第一节中的规则:
\begin{lstlisting}[caption={}]
calc : ("with" ident ("," ident)* ":")? expr ;
\end{lstlisting}

\item 该方法首先声明一些局部变量:
\begin{lstlisting}[caption={}]
AST *Parser::parseCalc() {
	Expr *E;
	llvm::SmallVector<llvm::StringRef, 8> Vars;
\end{lstlisting}

\item 要做的第一个决定是需不需要解析可选组。可选组以with标记开始,因此将标记与以下值进行比较:
\begin{lstlisting}[caption={}]
	if (Tok.is(Token::KW_with)) {
		advance();
\end{lstlisting}

\item 接下来,应该是一个标识符:
\begin{lstlisting}[caption={}]
	if (expect(Token::ident))
		goto _error;
	Vars.push_back(Tok.getText());
	advance();
\end{lstlisting}
如果是标识符,则将其保存在Vars向量中。否则,为一个句法错误,将被单独处理。

\item 在calc的语法中,当前位置后面可以跟着重复组,此时需要解析更多的标识符,重复组里标识符之间用逗号分隔:
\begin{lstlisting}[caption={}]
	while (Tok.is(Token::comma)) {
		advance();
		if (expect(Token::ident))
			goto _error;
		Vars.push_back(Tok.getText());
		advance();
	}
\end{lstlisting}
重复组以with开始,标记的测试成为while循环的条件,实现零或多次重复。循环内的标识符会和之前的处理一样。

\item 最后,可选组需要在末尾加冒号:
\begin{lstlisting}[caption={}]
	if (consume(Token::colon))
		goto _error;
	}
\end{lstlisting}

\item 现在,对expr规则进行解析:
\begin{lstlisting}[caption={}]
	E = parseExpr();
\end{lstlisting}

\item 通过调用,规则已经成功解析。现在,收集到的信息用于创建此规则的AST节点:
\begin{lstlisting}[caption={}]
	if (Vars.empty()) return E;
	else return new WithDecl(Vars, E);
\end{lstlisting}
\end{enumerate}

现在,只缺少错误处理代码了。检测句法错误很容易,但从错误中恢复却异常复杂。这里,使用了一种称为“恐慌模式”的简单方法。\par

恐慌模式下,将从标记流中删除标记,直到找到解析器可以使用的标记才继续工作。大多数编程语言都有表示结束的符号,例如:C++,可以使用\ ; (语句的结束) 或 \} (块的结束)。\par

另一方面,错误可能是正在寻找的符号不见了。这种情况下,许多标识可能在解析器继续之前就删除了,这并不像听起来那么糟糕。现今,编译器的速度非常重要。在出现错误的情况下,开发人员可以查看第一个错误消息,修复它,并重新启动编译器。这与使用穿孔卡片不同,在穿孔卡片中需要获取尽可能多的错误消息,因为编译器的下次运行只可能是在第二天了。\par

\hspace*{\fill} \par %插入空行
\textbf{错误处理}

不使用一些任意的标记,而使用另一组标记。对于每个非终止符,都有一组标识可以跟在那个终止符的后面,这是由那个终止符的语法规则规定的。\par

\begin{enumerate}
\item 如果是calc,跟在这个非终止符之后的只能是输入的结束。它的实现比较简单:
\begin{lstlisting}[caption={}]
_error:
	while (!Tok.is(Token::eoi))
		advance();
	return nullptr;
}
\end{lstlisting}

\item 其他解析方法也以类似的方式构造。parseExpr()是对expr规则的翻译:
\begin{lstlisting}[caption={}]
Expr *Parser::parseExpr() {
	Expr *Left = parseTerm();
	while (Tok.isOneOf(Token::plus, Token::minus)) {
		BinaryOp::Operator Op =
			Tok.is(Token::plus) ? BinaryOp::Plus :
								  BinaryOp::Minus;
		advance();
		Expr *Right = parseTerm();
		Left = new BinaryOp(Op, Left, Right);
	}
	return Left;
}
\end{lstlisting}
expr规则里的重复组会转换为while循环。注意,isOneOf()方法的使用简化了多标记情况下的检查。

\item term规则的编码看起来也一样:
\begin{lstlisting}[caption={}]
Expr *Parser::parseTerm() {
	Expr *Left = parseFactor();
	while (Tok.isOneOf(Token::star, Token::slash)) {
		BinaryOp::Operator Op =
		Tok.is(Token::star) ? BinaryOp::Mul :
		BinaryOp::Div;
		advance();
		Expr *Right = parseFactor();
		Left = new BinaryOp(Op, Left, Right);
	}
	return Left;
}
\end{lstlisting}
这个方法与parseExpr()非常地相似,你可能会想将它们组合成一个。在语法中,这是可能的,即用一条规则来处理乘法运算符和加法运算符。使用两个规则而不是一个规则的好处是,运算符的优先级与计算的数学顺序非常吻合。如果将两个规则结合起来,则需要在其它地方指出计算顺序。

\item 最后,需要实现factor规则:
\begin{lstlisting}[caption={}]
Expr *Parser::parseFactor() {
	Expr *Res = nullptr;
	switch (Tok.getKind()) {
		case Token::number:
			Res = new Factor(Factor::Number, Tok.getText());
			advance(); break;
\end{lstlisting}
这里使用switch语句似乎更合适,而不是使用if和else if,因为每个选项都只从一个标记开始。通常,应该考虑更偏爱哪种翻译模式。如果以后需要改变解析方法,那么在每个方法都有相同的语法规则实现方式时,就是一种优势。

\item 如果使用switch语句,在default情况下会进行错误处理:
\begin{lstlisting}[caption={}]
		case Token::ident:
			Res = new Factor(Factor::Ident, Tok.getText());
			advance(); break;
		case Token::l_paren:
			advance();
			Res = parseExpr();
			if (!consume(Token::r_paren)) break;
		default:
			if (!Res) error();
\end{lstlisting}
我们在这里避免发出错误消息。

\item 如果括号表达式中有句法错误,则会发出错误消息。这样可以避免第二个错误消息触发:
\begin{lstlisting}[caption={}]
			while (!Tok.isOneOf(Token::r_paren, Token::star,
								Token::plus, Token::minus,
								Token::slash, Token::eoi))
			advance();
	}
	return Res;
}
\end{lstlisting}

\end{enumerate}

这很简单,不是吗?当您记住了使用的模式,就会发现基于语法规则编写解析器很刻板。这种类型的解析器称为递归下降解析器。

\begin{tcolorbox}[colback=blue!5!white,colframe=blue!75!black, title=递归下降解析器不能适配所有语法]
语法必须满足某些条件才能适合构造递归下降解析器。这类语法叫做LL(1)。事实上,您可以在网上找到的大多数语法都不属于这一类语法。大多数关于编译器构造理论的书籍都解释了这一原因。关于这个主题的经典书籍是所谓的“龙书”,由Aho, Lam, Sethi和Ullman编写的《编译器:原理,技术和工具》。
\end{tcolorbox}


\hspace*{\fill} \par %插入空行
\textbf{抽象语法树}

解析的结果是AST,是输入程序的另一种紧凑表示形式。它捕获了重要的信息。许多编程语言都有作为分隔符的符号,而这些符号没有意义。例如,在C++中,分号;表示单个语句的结束。当然,这些信息对解析器很重要。当我们将语句转换为内存中的表示,分号就可以删除了。\par

如果您查看示例表达式语言的第一条规则,那么很明显with关键字、逗号、顿号和冒号:对程序的意义来说并不真正重要。重要的是可以在表达式中使用的声明变量列表。结果只需要两个类来记录信息:Factor保存数字或标识符,BinaryOp保存算术运算符和表达式的左右两边,WithDecl存储声明的变量和表达式的列表。AST和Expr仅用于创建普通类的层次结构。\par

除了来自解析过的输入的信息外,还支持在使用访问者模式时进行树形遍历。这些都在AST.h头文件中:\par


\begin{enumerate}
\item 从访问者接口开始:
\begin{lstlisting}[caption={}]
#ifndef AST_H
#define AST_H

#include "llvm/ADT/SmallVector.h"
#include "llvm/ADT/StringRef.h"

class AST;
class Expr;
class Factor;
class BinaryOp;
class WithDecl;

class ASTVisitor {
public:
	virtual void visit(AST &){};
	virtual void visit(Expr &){};
	virtual void visit(Factor &) = 0;
	virtual void visit(BinaryOp &) = 0;
	virtual void visit(WithDecl &) = 0;
};
\end{lstlisting}
访问者模式需要知道它必须访问的每个类。因为每个类也引用访问器,所以在文件的顶部声明所有类。请注意AST和Expr的visit()方法有一个默认实现,它什么也不做。

\item AST类在层次结构的根部:
\begin{lstlisting}[caption={}]
class AST {
public:
	virtual ~AST() {}
	virtual void accept(ASTVisitor &V) = 0;
};
\end{lstlisting}

\item 类似地,Expr是与表达式相关的AST根类:
\begin{lstlisting}[caption={}]
class Expr : public AST {
public:
	Expr() {}
};
\end{lstlisting}

\item Factor类存储一个数字或变量名:
\begin{lstlisting}[caption={}]
class Factor : public Expr {
public:
	enum ValueKind { Ident, Number };
	
private:
	ValueKind Kind;
	llvm::StringRef Val;
	
public:
	Factor(ValueKind Kind, llvm::StringRef Val)
	: Kind(Kind), Val(Val) {}
	ValueKind getKind() { return Kind; }
	llvm::StringRef getVal() { return Val; }
	virtual void accept(ASTVisitor &V) override {
		V.visit(*this);
	}
};
\end{lstlisting}
本例中,数字和变量的处理方法几乎相同,因此可以只创建一个AST节点类来表示它们,Kind成员告诉我们实例代表的是数字还是变量。在更复杂的语言中,通常希望有不同的AST类,例如:用于数字的NumberLiteral类和用于变量引用的VariableAccess类。

\item BinaryOp类保存了计算表达式所需的数据:
\begin{lstlisting}[caption={}]
class BinaryOp : public Expr {
public:
	enum Operator { Plus, Minus, Mul, Div };
	
private:
	Expr *Left;
	Expr *Right;
	Operator Op;
	
public:
	BinaryOp(Operator Op, Expr *L, Expr *R)
		: Op(Op), Left(L), Right(R) {}
	Expr *getLeft() { return Left; }
	Expr *getRight() { return Right; }
	Operator getOperator() { return Op; }
	virtual void accept(ASTVisitor &V) override {
		V.visit(*this);
	}
};	
\end{lstlisting}
与解析器相比,BinaryOp类没有区分乘法运算符和加法运算符。操作符的优先级隐含在树型结构中。

\item 最后,WithDecl存储声明的变量和表达式:
\begin{lstlisting}[caption={}]
class WithDecl : public AST {
	using VarVector =
					llvm::SmallVector<llvm::StringRef, 8>;
	VarVector Vars;
	Expr *E;
	
public:
	WithDecl(llvm::SmallVector<llvm::StringRef, 8> Vars,
			 Expr *E)
		: Vars(Vars), E(E) {}
	VarVector::const_iterator begin()
								{ return Vars.begin(); }
	VarVector::const_iterator end() { return Vars.end(); }
	Expr *getExpr() { return E; }
	virtual void accept(ASTVisitor &V) override {
		V.visit(*this);
	}
};
#endif	
\end{lstlisting}

\end{enumerate}

AST是在解析期间构造的。语义分析检查抽象语法树是否符合语言的含义(例如,使用的变量已经声明过了),并可能扩充抽象语法树。语义分析之后,抽象语法树就可以用来进行代码生成了。
